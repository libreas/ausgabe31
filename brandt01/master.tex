\documentclass[a4paper,
fontsize=11pt,
%headings=small,
oneside,
numbers=noperiodatend,
parskip=half-,
bibliography=totoc,
final
]{scrartcl}

\usepackage{synttree}
\usepackage{graphicx}
\setkeys{Gin}{width=.4\textwidth} %default pics size

\graphicspath{{./plots/}}
\usepackage[ngerman]{babel}
\usepackage[T1]{fontenc}
%\usepackage{amsmath}
\usepackage[utf8x]{inputenc}
\usepackage [hyphens]{url}
\usepackage{booktabs} 
\usepackage[left=2.4cm,right=2.4cm,top=2.3cm,bottom=2cm,includeheadfoot]{geometry}
\usepackage{eurosym}
\usepackage{multirow}
\usepackage[ngerman]{varioref}
\setcapindent{1em}
\renewcommand{\labelitemi}{--}
\usepackage{paralist}
\usepackage{pdfpages}
\usepackage{lscape}
\usepackage{float}
\usepackage{acronym}
\usepackage{eurosym}
\usepackage[babel]{csquotes}
\usepackage{longtable,lscape}
\usepackage{mathpazo}
\usepackage[normalem]{ulem} %emphasize weiterhin kursiv
\usepackage[flushmargin,ragged]{footmisc} % left align footnote
\usepackage{ccicons} 

\usepackage{listings}

\urlstyle{same}  % don't use monospace font for urls

\usepackage[fleqn]{amsmath}

%adjust fontsize for part

\usepackage{sectsty}
\partfont{\large}

%Das BibTeX-Zeichen mit \BibTeX setzen:
\def\symbol#1{\char #1\relax}
\def\bsl{{\tt\symbol{'134}}}
\def\BibTeX{{\rm B\kern-.05em{\sc i\kern-.025em b}\kern-.08em
    T\kern-.1667em\lower.7ex\hbox{E}\kern-.125emX}}

\usepackage{fancyhdr}
\fancyhf{}
\pagestyle{fancyplain}
\fancyhead[R]{\thepage}

%meta

%meta

\fancyhead[L]{S. Brandt \\ %author
LIBREAS. Library Ideas, 31 (2017). % journal, issue, volume.
\href{http://nbn-resolving.de/}
{}} % urn
\fancyhead[R]{\thepage} %page number
\fancyfoot[L] {\ccLogo \ccAttribution\ \href{https://creativecommons.org/licenses/by/3.0/}{\color{black}Creative Commons BY 3.0}}  %licence
\fancyfoot[R] {ISSN: 1860-7950}

\title{\LARGE{Entspannt euch!\\ Ein Kommentar zur Emotionalität in aktuellen bibliothekarischen Debatten}} % title
\author{Susanne Brandt} % author

\setcounter{page}{1}

\usepackage[colorlinks, linkcolor=black,citecolor=black, urlcolor=blue,
breaklinks= true]{hyperref}

\date{}
\begin{document}

\maketitle
\thispagestyle{fancyplain} 

%abstracts

%body
Über Entscheidungen und Äußerungen zu Bibliotheken kann man streiten --
öffentlich und leidenschaftlich. Im Juli 2016 hat ein Schild an der
Eingangstür der Stadtbücherei Nordenham die Gemüter erregt, eine lange
Kette von Facebook-Kommentaren ausgelöst\footnote{\url{https://www.facebook.com/StadtbuechereiNordenham/},
  19.7.2016} und den Medien reichlich Futter für das Sommerloch
geliefert. In Heft 10/2016 der Zeitschrift BuB\footnote{Buch und
  Bibliothek 10 (2016), S.592-593, ab Februar 2017 online:
  \url{http://b-u-b.de/archiv/}} wurden schließlich wichtige Aspekte der
Diskussion als Pro \& Contra dargestellt -- hier nun deutlich
begründeter und sachlicher. In der unmittelbaren Reaktion auf das Schild
\enquote{Achtung -- sie betreten eine Pokémon freie Zone} wurden andere
Töne angeschlagen: Mehrheitlich erregt, höhnisch, warnend,
verärgert\ldots{}

Ein halbes Jahr zuvor hatte in der Schweiz der Chef der ETH-Bibliothek
Rafael Ball mit einem Interview unter der Überschrift \enquote{Weg mit
den Büchern}\footnote{Neue Züricher Zeitung, 7.2.2016, Weg mit den
  Büchern. Interview Michael Furger mit Raphael Ball:
  \url{http://www.nzz.ch/nzzas/nzz-am-sonntag/bibliotheken-weg-mit-den-buechern-interview-rafael-ball-eth-ld.5093}}
für Aufruhr gesorgt, in dem er unter anderem auf die, seiner Meinung
nach, schwindende Bedeutung von Gemeindebüchereien eingeht. Die
Positionen scheinen konträr: Der eine hält Bücher in Bibliotheken für
verzichtbar, der andere lässt die Bücher wo sie sind, will jedoch keine
Pokémon-Spiele in der Bibliothek. Die Reaktionen auf diese öffentlichen
Äußerungen sind in vergleichbarer Weise heftig, emotional aufgeladen und
offenbar von ähnlichen Ängsten besetzt: Im Februar befürchteten viele
mit der Abschaffung der Buchbestände das Ende der Bibliotheks- und
Lesekultur. So heißt es in einem der Online-Kommentare, die sich an die
Online-Veröffentlichung des Beitrags anschließen:

\begin{quote}
\enquote{Wenn man den Chef der ETH-Bibliothek sagen hört,}Weg mit den
Büchern!\enquote{, ist dann die höchste Zeit ihn von diesem Amte zu
entlassen. (\ldots{}) Schade, dass auch Herr Dr.~Ball zum Opfer der
computerisierten Technologie des Lesens gefallen ist. Auf jeden Fall,
sind die im Artikel von Herrn Furger angeführten Aussagen irreführend
und unverzeihlich: die fördern das Analphabetentum.}\footnote{Ebd.}
\end{quote}

Auch zu dieser Debatte gab es einige Tage später auf dem Portal des
Schweizer Radio und Fernsehen einen sachlich gehaltenen Rückblick auf
das erregte Echo.\footnote{\url{http://www.srf.ch/kultur/literatur/buecher-sind-von-gestern-user-haben-eine-andere-meinung}}

Im Juli dann wurde ebenso das Ende der Bibliotheken befürchtet, sobald
dort nicht alles Digitale in seinen vielfältigen Spielarten ungehindert
Platz finden kann und darf. Und manche nahmen das Pokémon-Verbot gleich
als Anlass zur großen Rundum-Kritik -- wie in folgendem
Facebook-Kommentar-Beispiel:

\begin{quote}
\enquote{Während in den USA die normale Storytime schon mit Pokémon
Cosplay aufgelockert wird, töttern wir hier blöde rum. Macht mich
aggressiv. Mindestens so aggressiv wie Lehrern, Eltern und Kollegen die
Vorzüge von Graphic Novels und Comics zu erläutern}.\footnote{\url{https://www.facebook.com/StadtbuechereiNordenham/},
  19.7.2016}
\end{quote}

In beiden Debatten entwickelten sich im Verlauf der verzweigten
Kommentarlinien und Medienechos allerlei Verkürzungen der ursprünglichen
Aussagen, Missverständnisse, Anschuldigungen, Entrüstungen, Vorwürfe,
Vermischungen von Themen, die wenig miteinander zu tun haben.

Warum werden beide Diskussionen um provokante Aussagen derart emotional
geführt, in manchen Äußerungen auch deutlich herablassend, beleidigend
und aggressiv im Ton?

Meine erste These dazu lautet:

In der einen wie der anderen Diskussion finden Untergangsszenarien,
Ängste vor einem aus verschiedenen Gründen heraufbeschworenen
Zurückbleiben oder gar Verschwinden der Bibliothek ihren Ausdruck. In
beiden Diskussionen drückt sich für mich eine Verunsicherung oder
Verärgerung aus, die den Ton und die Strategien der Verteidigung so
scharf werden lässt.

Was dabei vermutlich alle wissen und erfahren: Das konkurrenzlose
Merkmal bisheriger Bibliothekskonzepte, per Ausleihe etwas Stoffliches
und damit etwas Standortbezogenes bereitzustellen, was sonst nur
käuflich zu haben wäre, hat durch die mobile und in vielen Fällen quasi
\enquote{freie} Verfügbarkeit digitaler Medien und Informationen einen
anderen Stellenwert bekommen. Bei der Beantwortung vieler Alltagsfragen
ist das Internet bibliotheksunabhängig längst die erste Wahl und ergänzt
beziehungsweise ersetzt als Auskunfts- und Unterhaltungsmedium die
bislang klassischen Ausleihangebote von Bibliotheken in vielen
Bereichen. Das ist so und das wird sich weiter verstärken. Damit
verbunden hat sich über soziale Netze eine Gewohnheit des Tauschens und
Teilens entwickelt, die gegenüber dem klassischen \enquote{Teilen per
Ausleihe} in mancher Hinsicht deutliche Vorteile durch mehr Mobilität,
Partizipation und Flexibilität bietet. Diese gravierenden Veränderungen
erleben all jene, die diese Entwicklung mit einer Erweiterung digitaler
Angebote in Bibliotheken beantworten und begleiten ebenso wie all jene,
die sich eher kritisch mit digitalen Trends auseinandersetzen, als
fachliche wie emotionale Herausforderung, geht es doch um nichts
geringeres als um eine gravierende Neuorientierung in der eigenen Arbeit
und des von Menschen gestalteten Angebotes unter dem Druck einer immer
noch dominanten, oft eher an den \enquote{alten} Aufgaben ausgerichteten
Steigerungslogik und Erfolgserwartung seitens der Träger.

Bemerkenswert ist für mich, dass im Unterschied dazu im Buchhandel eher
ein anderer Umgang mit dem medialen Wandel zu beobachten ist. Natürlich
ist hier eine Vergleichbarkeit nur bedingt gegeben. Interessant scheint
mir in diesem Zusammenhang jedoch, dass offenbar gerade die Sortimenter
mit deutlich erkennbarem, gern auch eigenwilligem Profil eher Erfolge zu
verzeichnen haben als die großen Buchkaufhäuser. Auch für den
stationären Buchhandel hat sich durch den bequemen Online-Vertrieb
großer Anbieter eine massive Konkurrenzsituation ergeben. Also bietet er
genau das, was es online eben nicht gibt: Atmosphäre, die Präsenz einer
bewussten Auswahl, ein unverwechselbares Gesicht -- und nicht selten den
Mut, auch solche Themen zu inszenieren, die nicht überall zu finden
sind. Daneben bleibt natürlich die Bestellmöglichkeit für alle anderen
Dinge. Aber das, was den Menschen, die in der Buchhandlung arbeiten, am
Herzen liegt -- um hier bewusst eine emotionale Formulierung zu
verwenden - darf auch sichtbar werden. Für viele Kunden scheint genau in
dieser atmosphärischen und emotional-menschlichen Ausstrahlung der Reiz
zu liegen: Offen und einladend präsentiert sich dieser Ort, aber nicht
beliebig.

Auch bei den umstrittenen Standpunkten der oben zitierten
bibliothekarischen Diskussionen haben Menschen, die für die Arbeit einer
Einrichtung Verantwortung tragen, zum Ausdruck gebracht, was ihnen
\enquote{am Herzen liegt}, haben neben allen sachlichen Argumenten, die
dabei eine Rolle spielen, zugleich persönliche Einschätzungen und Ziele
erkennen lassen.

Wer das tut, macht sich angreifbar. Was bei diesem Vergleich
bemerkenswert ist: Schnell werden solche Positionen, die sich beim
Beispiel aus Nordenham deutlich auf eine konkrete Einrichtung beziehen,
als Angriff und Provokation für den gesamten Berufsstand und den
gesellschaftlichen Auftrag gewertet.

Vor dem Hintergrund solcher Beobachtungen lautet meine zweite These, die
sich vor allem auf die Äußerungen zu Nordenham von Fachkolleginnen und
-kollegen bezieht:

Mit der mitunter so erregt bis aggressiv geprägten Streitkultur um den
vermeintlich richtigen Zukunftskurs von Bibliotheken offenbart sich der
alte Kampf gegen ein ungeliebtes Image, das viele im Zuge des digitalen
Wandels nun endlich gegen ein modernes und \enquote{cooles} Image
eintauschen möchten. Dieses oft als \enquote{verstaubt} abgewertete
Image scheint noch immer ein wunder Punkt zu sein, und entsprechend
empfindlich fallen viele Reaktionen auf Entscheidungen aus, die einer
erhofften Aufwertung im Wege stehen könnten.

Plausibel erscheint mir diese These u.a. aufgrund von entsprechenden
Äußerungen wie folgendes Beispiel aus der Facebook-Diskussion:

\begin{quote}
\enquote{Ich denke, dass das daran liegt, dass durch eine solche Aktion
ein veraltetes Bild von Bibliotheken wieder heraufbeschworen wird. Und
es gibt so viele Bibliotheken, die sich alle Mühe geben, dieses Bild
loszuwerden. Wenn dann ein solcher Post durch die sozialen Netzwerke
geht, ist da die Befürchtung, dass das einen Rückschritt bedeutet und
das ist dann natürlich sehr frustrierend}\footnote{Ebd.}
\end{quote}

Unklar und verschwommen bleibt bei so ausgedrückten Befürchtungen,
welches \enquote{neue Bild} denn das \enquote{alte} und ungeliebte
konkret ersetzen soll. Vielleicht macht das die Diskussion in vielen
Teilen so aggressiv: Weil gegen etwas gekämpft wird, ohne eine
umfassende Vision oder Zielvorstellung von dem zu haben, was das
\enquote{andere} in allen seinen Dimensionen ausmacht. Zumindest scheint
es in der Kürze der eifrigen Kurzkommentare schwer zu sein, von einer
oberflächlichen Erregung zu einem fairen Diskurs über verschiedene
Zukunftsmodelle zu gelangen.

Mag sein, dass Bibliotheken in der Zukunft kleiner werden, dass sie neue
und veränderte Aufgaben in der Medienvermittlung übernehmen und weiter
ausbauen, ihre Räumlichkeiten anders gestalten, mit diesen wie mit jenen
Medien ihre Regale füllen. Mag sein, dass sie neue, anders messbare
Qualitäten ausbilden, wenn sie sich mutiger und konsequenter einer
Steigerungslogik widersetzen, die noch immer maßgeblich durch Ausleih-
und Besucherzahlen gesteuert wird. Solange vor allem in öffentlichen
Bibliotheken angespannt darum gerungen wird, wie digitale Angebote dort
ihren Platz finden, solange diese Angebote nicht an einem erkennbaren
Profil oder Ziel ausgerichtet werden, das eine Haltung deutlich macht
und gern auch zur Diskussion stehen kann, solange ist es schwierig,
wieder zu einer entspannteren Wahrnehmung von solidarisch zu
verstehenden Aufgaben und mehrdimensionalen Visionen zurückzukehren.

Umso nötiger wäre es meines Erachtens, Besonnenheit und Respekt gerade
auch dann im Spiel zu lassen, wenn Kollegen umstrittene Positionen
beziehen und den Mut haben, diese öffentlich zu begründen, um damit das
Weiterdenken in Bewegung bringen.

Meine dritte und letzte These an dieser Stelle lautet deshalb:

Ein fair und respektvoll geführter Diskurs könnte mehr in Bewegung
bringen, als kurzzeitig auftauchende und wieder verschwindende Pokémons
das tun. Der Diskurs könnte so vielleicht auch jene spielerische
Komponente wiedergewinnen, die im beschriebenen Beispiel zwar mehrfach
beschworen wurde, dem streckenweise sehr verbissen geführten Austausch
aber gleichzeitig abhanden gekommen ist -- und dem so genannten Spiel,
um das es ging, wohl auch.

Denn wer mit dem Pokémon-Spiel die Vorstellung verbindet, Kinder so in
die Bücherei zu locken, hat sich bereits vom Grundprinzip des zweckfrei
Spielerischen verabschiedet. Wo ein Spiel mit eigennützigen Absichten
verbunden wird, ist es im Grunde kein Spiel mehr, sondern Teil einer
gezielten Werbestrategie, die dann auch als solche benannt werden
sollte.

Gestritten wird hier also nach meinem Eindruck nicht um die Frage nach
dem Recht auf Spiel in der Bibliothek, sondern um Imagegewinn, um das
durchaus verständliche Interesse, im eigenen, vielleicht neu
geschaffenen Berufsbild nicht verunsichert oder hinterfragt zu werden.

Deshalb:

Entspannt euch! Und wenn mal wieder darüber gestritten werden kann, was
Mitarbeitende und Besucher in Bibliotheken tun dürfen, sollen, müssen
oder auch nicht, lässt sich der Diskurs dann vielleicht spielerischer
führen: mit Fantasie, Kreativität und Offenheit im Wechselspiel der
verschiedenen Denkmöglichkeiten\ldots{}

Mein Wunsch zum Schluss wäre, dass wieder mehr Mut und Respekt statt
Abwertung und Ängstlichkeit Einzug halten mögen in unsere
Fachdiskussionen. Denn mit Mut und Respekt lässt sich offener und
kritischer wahrnehmen und einschätzen, wie und warum andere Kolleginnen
und Kollegen an anderen Orten andere Entscheidungen treffen, andere
Visionen haben und andere Profile ihrer Arbeit mit Leben und Ideen
füllen.

Entspannt euch -- und fürchtet euch nicht vor der Vielfalt!

<<<<<<< Updated upstream
\begin{center}\rule{0.5\linewidth}{\linethickness}\end{center}

=======
%autor
>>>>>>> Stashed changes
\textbf{Susanne Brandt}, geb.1964 in Hamburg, Studium in
Bibliothekswesen und Kulturwissenschaften, Qualifikation
Rhythmisch-musikalische Erziehung und bibliotherapeutische
Weiterbildung, seit 1995 zahlreiche Buchveröffentlichungen und Beiträge
in Zeitschriften und Anthologien; ab 1987 Leiterin der Musikbibliothek
in Cuxhaven, ab 2000 Bibliotheksleiterin in
Westoverledingen/Ostfriesland, seit Juni 2011 Lektorin bei der
Büchereizentrale Schleswig-Holstein in Flensburg.

\end{document}